\documentclass{article}
\usepackage{appendix}
\usepackage{graphicx}
\usepackage{pdfpages}
\usepackage{fullpage}
\usepackage{hyperref}

\title{Architecture Reconstruction: Zeegu-React \\ \normalsize Software Architecture, MSc (Spring 2024) \\ KSSOARC2KU \\}
\author{Bjarke Brodin Larsen -- bjal@itu.dk}


\begin{document}
\maketitle
\tableofcontents

\clearpage
\section{Introduction}

\href{https://zeeguu.com}{Zeegu} is a language learning assistance tool
that helps users learn by reading and revising previously read language on a word-by-word basis.
It contains serveral features such as tracking progress and providing material for language learning
with on demand translations. 
It is also meant to offer some functionality to teachers interacting with the material provided,
but I have not probed this part of the app much and leave this mostly out of scope.

\vspace*{1em}\noindent
The front-end of Zeegu is the subject matter of this paper.
More specifically, I attempt to recover information about its architecture,
and subsequently reflect on the results of said recovery.

\section{Methodology: tooling and process step-by-step}

In the following section I will describe which tools I 
deployed in what order and why I chose to do so.

\subsection{Getting acquainted with the system -- ChatGPT}

Because getting hints and ideas about the elephant we are attempting to 
map out is not dependent on precision and in the beginning is mostly a
text-parsing task, I utilized ChatGPT\cite{gpt4} to summarize the
contents of the repository and get an introductory idea 
(see Appendix \ref{apx:a}).
Because ChatGPT can search the web it surprised me by being adept at 
suggesting tools for further inspection which might have come in handy!

\subsection{Getting a more exact feel of function -- running the system}

After quickly getting some hints and a notion about what system I am dealing with,
I proceed to run the system using the online beta deployment hosted at \href{https://zeeguu.com}{https://zeeguu.com}.
This is mostly to probe the system functionality and improve my mental map of how the codebase manifests to the user.

\subsection{Researching and briefly parsing the code manually}

By reviewing and researching the React framework on which the subject codebase is built,
and subsequently simply looking at the code. 
I attempt to acquire a feel for how the code fits together,
how are string translations handled,
how are components themed and global variables managed,
how is routing done.

\subsection{Mining data and producing a module view -- python3, networkx, pyvis}
\label{mining}

With a pretty good idea about the overall system structure,
I begin semi-automatically attempting to map the codebase using python.
Code can be found at \href{https://github.com/bjarkebrodin/recovery}{https://github.com/bjarkebrodin/recovery}.
I do this by first extracting the imports between files (see Figure \ref{fig:d:raw})
then subsequently I abstract to top-level modules and consider only imports in between such modules (see Figure \ref{fig:d:abs} and Figure \ref{fig:d:clean}).
This approach is very similar to the one demonstrated by Mircea Lungo in the reconstruction lectures\cite{reconstruction}.
These networks are visualized dynamically as html files and the reader can inspect them at will from the repository,
to make the data inspectable they are programmed with tooltips that show detailed information about nodes (see Figure \ref{fig:d:nodetooltip}), and edges (see Figure \ref{fig:d:edgetooltip}).
I visualize the gathered data and attempt to infer interesting facts about the architecture.
The outcome of this step is described in more detail throughout the remainder of this paper.
I use the tools \verb|pyvis|\cite{pyvis} and \verb|networkx|\cite{networkx} to help me do this without implementing graphs and visualisation myself.

\subsection{Superficially assesing external dependency health -- npm audit}

As a finishing step, only because it is so easily available in npm projects:
to quickly ascertain the health of external dependencies used I run \verb|npm audit|
to get an idea about how much (potential) debt there might be in the way external dependencies are used.



\section{Results}

Through semi-automatically eliciting an approximate dependency (JavaScript \verb|import|) structure of the codebase
(as described in Section \ref{mining}),
I produce a module view of the Zeegu front-end (see Figure \ref{fig:3:clean}).

\begin{figure}[h]
  \centering 
  \includegraphics[width=\textwidth]{graphics/module_abstraction_cleaned.png}
  \caption{Cleaned and colored visualisation of all import statements abstracted to top-level modules. 
  Thicker edges represent more imports (logarithmically scaled) and edge 
  direction represents which what is being imported to where.
  Colors do not have meaning but are meant to help better decode the network by eye.
  Modules are appended with information about how many times they are imported and how many times they import another module.
  A bigger circle means a module is imported more times by other modules (\textit{out} value),
  i.e. might be important or central to the architecture. \\ \vspace*{.25em} \\
  \hspace*{2em}\textit{out} -- times imported in a different top-level module \\
  \hspace*{2em}\textit{in} -- times importing a from different top-level module}
  \label{fig:3:clean}
\end{figure}

To frame the information presented in Figure \ref{fig:3:clean} and prime the reader 
for a brief discussion of this data (Section \ref{discussion}),
I begin by briefly outlining and summarizing the degree of interest of some of the most referenced/central modules,
in Section \ref{discussion} I go on consider some of the most interesting areas of this list.

\begin{itemize}
  \item \verb|utils| -- A collection of project local utility code.
                        It has no dependency on other internal modules, 
                        however, 
                        there may be something to be said about
                        its submodules \verb|utils.routing| and \verb|utils.cookies|.
                        % for discussion: rather opaque, suggest routing and cookies as top level modules
  \item \verb|hooks| -- A module of various react hooks supporting ad-hoc stateful component enhancement.
                        This is slightly interesting because somewhat random functionality seems grouped together,
                        if nothing else only based on a react convention to have a module to dump all the hoooks in.
                        % for discussion: hooks could be kept closer to what they provide functions to, keep code that changes together close together!
                        % for discussion: grouped together based on which react abstraction they use, not what they logically belong to
  \item \verb|contexts| -- Much like the \verb|hooks| module, this module contains injectable "contexts", 
                           a react specific way to do inversion of dependency with a provider/consumer paradigm.
  \item \verb|components| -- (Supposedly) A module of the general UI components used in the frontend,
                             at first glance it is very interesting that components seems to import from both \verb|reader|, \verb|i18n| and \verb|exercises| -- there are also 2 imports from \verb|utils| but this is just common code so not of concern.
                             % for discussion: criss-cross dependencies, intercoupling
  \item \verb|PrivateRoute|, \verb|MainAppRouter|, \verb|PrivateRouteWithSidebar| -- Modules for navigating the different parts of the site,
                                                                                     these should arguably be encapsulated in a \verb|routing|
                                                                                     top-level module and could then be decoupled by fetching routes by lookup instead of reference. It is however not that bad to couple routing tightly as it is mostly expected in a react app of this size and state in development.
  \item \verb|i18n| -- This module contains the configuration of the \textbf{i18n} module for string internationalization,
                       meaning mostly a collection of strings in different languages to be used elsewhere.
                       Since it only exports and doesn't import from any other local module this is mostly uninteresting,
                       however, there may have been a benefit to adding a layer of indirection between how translations are implemented
                       and the abstraction of fetching strings by lookup. There is also a bit of funny business regarding how non-global
                       language is handled, it seems to be somewhat scattered around the codebase.
  \item \verb|articles| -- This module contains code related to showing, finding and working with articles on the Zeegu platform.
                           It exports to only \verb|teacher| and routers making it seem relatively well-encapsulated.
  \item \verb|words| -- Contains code for the words page and functionality for working with words.
                        The module is relatively uninteresting except for the fact that it has cross-dependencies with \verb|exercises|.
  \item \verb|teacher| -- This module contains functionality for teachers, 
                          it almost exclusively imports from \verb|i18n| and \verb|components|
                          and is only imported in routers, making it quite uninteresting.
  \item \verb|pages| -- This module contains the markup and some components used for some of the pages of the app,
                        not all though, some pages are in \verb|landingPage| and again some other pages are elsewhere.
  \item \verb|reader| -- This module implements the article reader and is thus a quite central part of the app,
                         it is however at first glance quite well decoupled and logically encapsulated nicely,
                         meaning initially, not very problematic or interesting structurally.
  \item \verb|assorted| -- This module contains miscellaneous utility code and a single hook,
                           it is opaque to me why this code is not in \verb|utils| and \verb|hooks| instead.
\end{itemize}


Note that to limit the scope I have somewhat arbitrarily elected a size/interest cutoff
of which top-level modules to include in this list,
otherwise there would simply be too much to cover.
Even discussing all the modules I list above in more depth would be relatively plentiful,
thus, I cover only some of the more appearant and superficial concerns in the following section.


\section{Discussion}
\label{discussion}

From visualising and inspecting the data more closely (see Figure \ref{fig:3:clean}), 
both by looking at the overview but also by interacting with 
\href{https://github.com/bjarkebrodin/recovery/blob/master/top_lvl_imports.html}{the generated html}
to inspect exactly what is imported in which module,
I consider a few aspects of the system.

\vspace*{1em}
\noindent
Starting off,
there seems to be several top level modules depending on eachother indicating a tighter degree of coupling
in between them than we would ideally like:
\begin{itemize}
  \item \verb|exercises| $\leftrightarrow$ \verb|components|
  \item \verb|exercises| $\leftrightarrow$ \verb|words|
  \item \verb|reader| $\leftrightarrow$ \verb|components|
  \item \verb|routers| $\leftrightarrow$ \verb|*| (transitively)
\end{itemize}

\vspace*{1em}
\noindent
Additionally,
it seems less that ideal at first glance that some utility code is in \verb|utils| and some other utility code is in \verb|assorted|,
there might be an intention behind this but it would also be explainable by different contributers using different modules for their
helper-code (I did not check this in the git files).
Either way it seems easy to refactor this to enhance overall structure.


\vspace*{1em}
\noindent
The \verb|hooks| module seems to group code by which category of react abstraction it uses.
While this is probably quite conventional,
I would argue that there would be a point to having hooks implemented closer to the code that they use.
As an example of my point,
\verb|hooks.useExtensionCommunication| is essentially just a hook-based interface for the code implemented in
\verb|utils.extension.extensionCommunication|,
thus I would consider moving the hook to \verb|utils.extension.useExtensionCommunication| (and similarly for other hooks).
This would clean the top-level scope and would make sense since when crawling backwards from where the hook is imported
would bring you straight to both files, potentially making it less opaque to debug and understand.


\vspace*{1em}
Many more things could be said about the system,
but for the sake of brevity I shall end my considerations here.
Overall,
even though a few top-level modules are tightly coupled,
the coupling does not seem particularly tight which is positive.
Appropriate react abstractions seem used reasonably well (hooks and contexts)
and translations are injected from a single point (even if the coupling to the \textbf{i18n} library is a bit tight).
The top-level structure could be more intentional,
providing developers with a better idea of how new code is categorized 
and assistance in forming a mapping between the visual/navigable structure of the app and the structure of the code.
By this I mean a clear way to guess exactly where to find the code responsible for rendering what is currently visible.





% what would we guess the main role of the front-end architecture is?
% - easily add new routes?
% - easily iterate and develop: modifiability?
% - easily collaborate: open source?
% is component theme coherent across different components?
% is component data-interface separated from component visuals?
% can components be changed without modifying where they appear?
% i18n coupled heavily to codebase?
% seems logically well-grouped but not structurally well decoupled, expand
% - many top-level modules have cross-dependencies, largely not structured as suppliers/consumers
% - better import hierarchy might be nice
% How coupled is routing to pages themselves?
% - could we do this better?


% A lot of the structure seems very makeshift and iteratively put together 
% without much architectural intention (beyond contexts, hooks, translations).
% Some work could be done to decouple and streamline inter-module dependencies,
% making it easier to develop and potentially version components separately as the codebase grows!



\vspace*{1em}
\noindent
I would have liked to produce some metrics, 
especially link depth and cyclomatic complexity (as ChatGPT suggested, see Appendix \ref{apx:a} p.6) 
but this did not work at first attempt and I was unable to invest sufficient time in getting working.
This would have been an interesting addition to the import structure!
I suspect especially cyclomatic complexity would be quite high 
because of the degree of coupling suggested by the module view.

\clearpage
\bibliography{lit}
\bibliographystyle{ieeetran}

\clearpage
\appendix
\section{ChatGPT Transcript}
\label{apx:a}
\begin{figure}[h]
\includepdf[width=\textwidth,pages=1]{appendix/chatgpt.pdf}
\end{figure}
\includepdf[width=\textwidth,pages=2-]{appendix/chatgpt.pdf}

\clearpage
\section{Approximate Time Allocation}
\label{apx:c}
\begin{itemize}
  \item Development \& tool learning: 55\%
  \item Researching react: 5\%
  \item Playing with/wrangling data: 10\%
  \item Reflecting on results: 5\%
  \item Writing: 25\%
\end{itemize}

\clearpage
\section{Data Visualisation}
\label{apx:d}

\begin{figure}[h]
\includegraphics[width=\textwidth]{graphics/raw_imports.png}
\caption{Unfiltered visualisation of all import statements in the sourcecode}
\label{fig:d:raw}
\end{figure}

\begin{figure}[h]
\includegraphics[width=\textwidth]{graphics/module_abstraction_dirty.png}
\caption{Visualisation of all import statements abstracted to top-level modules. 
Thicker edges represent more imports (logarithmically scaled) and edge 
direction represents which what is being imported to where.}
\label{fig:d:abs}
\end{figure}


\begin{figure}[h]
\includegraphics[width=\textwidth]{graphics/module_abstraction_cleaned.png}
\caption{Cleaned and colored visualisation of all import statements abstracted to top-level modules. 
Thicker edges represent more imports (logarithmically scaled) and edge 
direction represents which what is being imported to where.
modules are appended with information about how many times they are imported and how many times they import another module: \\
\hspace*{2em}\textit{out} -- times imported in a different top-level module \\
\hspace*{2em}\textit{in} -- times importing a from different top-level module}
\label{fig:d:clean}
\end{figure}

\begin{figure}[h]
\centering
\includegraphics[]{graphics/edge_tooltip.png}
\caption{Tooltip displayed on edges when visualisation is inspected dynamically as .html\\
\hspace*{2em}\textit{out} -- times imported in a different top-level module \\
\hspace*{2em}\textit{in} -- times importing a from different top-level module}
\label{fig:d:edgetooltip}
\end{figure}

\begin{figure}[h]
\centering
\includegraphics[]{graphics/node_tooltip.png}
\caption{Tooltip displayed on nodes when visualisation is inspected dynamically as .html\\
\hspace*{2em}\textit{out} -- times imported in a different top-level module \\
\hspace*{2em}\textit{in} -- times importing a from different top-level module}
\label{fig:d:nodetooltip}
\end{figure}





\end{document}